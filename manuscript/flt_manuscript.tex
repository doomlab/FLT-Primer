\documentclass[man]{apa6}
\usepackage{lmodern}
\usepackage{amssymb,amsmath}
\usepackage{ifxetex,ifluatex}
\usepackage{fixltx2e} % provides \textsubscript
\ifnum 0\ifxetex 1\fi\ifluatex 1\fi=0 % if pdftex
  \usepackage[T1]{fontenc}
  \usepackage[utf8]{inputenc}
\else % if luatex or xelatex
  \ifxetex
    \usepackage{mathspec}
  \else
    \usepackage{fontspec}
  \fi
  \defaultfontfeatures{Ligatures=TeX,Scale=MatchLowercase}
\fi
% use upquote if available, for straight quotes in verbatim environments
\IfFileExists{upquote.sty}{\usepackage{upquote}}{}
% use microtype if available
\IfFileExists{microtype.sty}{%
\usepackage{microtype}
\UseMicrotypeSet[protrusion]{basicmath} % disable protrusion for tt fonts
}{}
\usepackage{hyperref}
\hypersetup{unicode=true,
            pdftitle={A practical primer on processing semantic property norm data},
            pdfauthor={Erin M. Buchanan, Simon De Deyne, \& Maria Montefinese},
            pdfkeywords={semantic, property norm task, tutorial},
            pdfborder={0 0 0},
            breaklinks=true}
\urlstyle{same}  % don't use monospace font for urls
\usepackage{color}
\usepackage{fancyvrb}
\newcommand{\VerbBar}{|}
\newcommand{\VERB}{\Verb[commandchars=\\\{\}]}
\DefineVerbatimEnvironment{Highlighting}{Verbatim}{commandchars=\\\{\}}
% Add ',fontsize=\small' for more characters per line
\usepackage{framed}
\definecolor{shadecolor}{RGB}{248,248,248}
\newenvironment{Shaded}{\begin{snugshade}}{\end{snugshade}}
\newcommand{\AlertTok}[1]{\textcolor[rgb]{0.94,0.16,0.16}{#1}}
\newcommand{\AnnotationTok}[1]{\textcolor[rgb]{0.56,0.35,0.01}{\textbf{\textit{#1}}}}
\newcommand{\AttributeTok}[1]{\textcolor[rgb]{0.77,0.63,0.00}{#1}}
\newcommand{\BaseNTok}[1]{\textcolor[rgb]{0.00,0.00,0.81}{#1}}
\newcommand{\BuiltInTok}[1]{#1}
\newcommand{\CharTok}[1]{\textcolor[rgb]{0.31,0.60,0.02}{#1}}
\newcommand{\CommentTok}[1]{\textcolor[rgb]{0.56,0.35,0.01}{\textit{#1}}}
\newcommand{\CommentVarTok}[1]{\textcolor[rgb]{0.56,0.35,0.01}{\textbf{\textit{#1}}}}
\newcommand{\ConstantTok}[1]{\textcolor[rgb]{0.00,0.00,0.00}{#1}}
\newcommand{\ControlFlowTok}[1]{\textcolor[rgb]{0.13,0.29,0.53}{\textbf{#1}}}
\newcommand{\DataTypeTok}[1]{\textcolor[rgb]{0.13,0.29,0.53}{#1}}
\newcommand{\DecValTok}[1]{\textcolor[rgb]{0.00,0.00,0.81}{#1}}
\newcommand{\DocumentationTok}[1]{\textcolor[rgb]{0.56,0.35,0.01}{\textbf{\textit{#1}}}}
\newcommand{\ErrorTok}[1]{\textcolor[rgb]{0.64,0.00,0.00}{\textbf{#1}}}
\newcommand{\ExtensionTok}[1]{#1}
\newcommand{\FloatTok}[1]{\textcolor[rgb]{0.00,0.00,0.81}{#1}}
\newcommand{\FunctionTok}[1]{\textcolor[rgb]{0.00,0.00,0.00}{#1}}
\newcommand{\ImportTok}[1]{#1}
\newcommand{\InformationTok}[1]{\textcolor[rgb]{0.56,0.35,0.01}{\textbf{\textit{#1}}}}
\newcommand{\KeywordTok}[1]{\textcolor[rgb]{0.13,0.29,0.53}{\textbf{#1}}}
\newcommand{\NormalTok}[1]{#1}
\newcommand{\OperatorTok}[1]{\textcolor[rgb]{0.81,0.36,0.00}{\textbf{#1}}}
\newcommand{\OtherTok}[1]{\textcolor[rgb]{0.56,0.35,0.01}{#1}}
\newcommand{\PreprocessorTok}[1]{\textcolor[rgb]{0.56,0.35,0.01}{\textit{#1}}}
\newcommand{\RegionMarkerTok}[1]{#1}
\newcommand{\SpecialCharTok}[1]{\textcolor[rgb]{0.00,0.00,0.00}{#1}}
\newcommand{\SpecialStringTok}[1]{\textcolor[rgb]{0.31,0.60,0.02}{#1}}
\newcommand{\StringTok}[1]{\textcolor[rgb]{0.31,0.60,0.02}{#1}}
\newcommand{\VariableTok}[1]{\textcolor[rgb]{0.00,0.00,0.00}{#1}}
\newcommand{\VerbatimStringTok}[1]{\textcolor[rgb]{0.31,0.60,0.02}{#1}}
\newcommand{\WarningTok}[1]{\textcolor[rgb]{0.56,0.35,0.01}{\textbf{\textit{#1}}}}
\usepackage{graphicx,grffile}
\makeatletter
\def\maxwidth{\ifdim\Gin@nat@width>\linewidth\linewidth\else\Gin@nat@width\fi}
\def\maxheight{\ifdim\Gin@nat@height>\textheight\textheight\else\Gin@nat@height\fi}
\makeatother
% Scale images if necessary, so that they will not overflow the page
% margins by default, and it is still possible to overwrite the defaults
% using explicit options in \includegraphics[width, height, ...]{}
\setkeys{Gin}{width=\maxwidth,height=\maxheight,keepaspectratio}
\IfFileExists{parskip.sty}{%
\usepackage{parskip}
}{% else
\setlength{\parindent}{0pt}
\setlength{\parskip}{6pt plus 2pt minus 1pt}
}
\setlength{\emergencystretch}{3em}  % prevent overfull lines
\providecommand{\tightlist}{%
  \setlength{\itemsep}{0pt}\setlength{\parskip}{0pt}}
\setcounter{secnumdepth}{0}
% Redefines (sub)paragraphs to behave more like sections
\ifx\paragraph\undefined\else
\let\oldparagraph\paragraph
\renewcommand{\paragraph}[1]{\oldparagraph{#1}\mbox{}}
\fi
\ifx\subparagraph\undefined\else
\let\oldsubparagraph\subparagraph
\renewcommand{\subparagraph}[1]{\oldsubparagraph{#1}\mbox{}}
\fi

%%% Use protect on footnotes to avoid problems with footnotes in titles
\let\rmarkdownfootnote\footnote%
\def\footnote{\protect\rmarkdownfootnote}


  \title{A practical primer on processing semantic property norm data}
    \author{Erin M. Buchanan\textsuperscript{1}, Simon De Deyne\textsuperscript{2}, \& Maria Montefinese\textsuperscript{3}}
    \date{}
  
\shorttitle{Processing Norms}
\affiliation{
\vspace{0.5cm}
\textsuperscript{1} Harrisburg University of Science and Technology\\\textsuperscript{2} University of Melbourne\\\textsuperscript{3} University of Padua}
\keywords{semantic, property norm task, tutorial}
\usepackage{csquotes}
\usepackage{upgreek}
\captionsetup{font=singlespacing,justification=justified}

\usepackage{longtable}
\usepackage{lscape}
\usepackage{multirow}
\usepackage{tabularx}
\usepackage[flushleft]{threeparttable}
\usepackage{threeparttablex}

\newenvironment{lltable}{\begin{landscape}\begin{center}\begin{ThreePartTable}}{\end{ThreePartTable}\end{center}\end{landscape}}

\makeatletter
\newcommand\LastLTentrywidth{1em}
\newlength\longtablewidth
\setlength{\longtablewidth}{1in}
\newcommand{\getlongtablewidth}{\begingroup \ifcsname LT@\roman{LT@tables}\endcsname \global\longtablewidth=0pt \renewcommand{\LT@entry}[2]{\global\advance\longtablewidth by ##2\relax\gdef\LastLTentrywidth{##2}}\@nameuse{LT@\roman{LT@tables}} \fi \endgroup}


\DeclareDelayedFloatFlavor{ThreePartTable}{table}
\DeclareDelayedFloatFlavor{lltable}{table}
\DeclareDelayedFloatFlavor*{longtable}{table}
\makeatletter
\renewcommand{\efloat@iwrite}[1]{\immediate\expandafter\protected@write\csname efloat@post#1\endcsname{}}
\makeatother
\usepackage{lineno}

\linenumbers

\authornote{Add complete departmental affiliations for each author here. Each new line herein must be indented, like this line.

Enter author note here.

Correspondence concerning this article should be addressed to Erin M. Buchanan, 326 Market St., Harrisburg, PA 17101. E-mail: \href{mailto:ebuchanan@harrisburgu.edu}{\nolinkurl{ebuchanan@harrisburgu.edu}}}

\abstract{
Semantic property listing tasks require participants to generate short propositions (e.g., \textless{}barks\textgreater{}, \textless{}has fur\textgreater{}) for a specific concept (e.g., dog). This task is the cornerstone of the creation of semantic property norms which are essential for modelling, stimuli creation, and understanding similarity between concepts. However, despite the wide applicability of semantic property norms for a large variety of concepts across different groups of people, the methodological aspects of the property listing task have received less attention, even though the procedure and processing of the data can substantially affect the nature and quality of the measures derived from them. The goal of this paper is to provide a practical primer on how to collect and process semantic property norms. We will discuss the key methods to elicit semantic properties and compare different methods to derive meaningful representations from them. This will cover the role of instructions and test context, property pre-processing (e.g., lemmatization), property weighting, and relationship encoding using ontologies. With these choices in mind, we propose and demonstrate a processing pipeline that transparently documents these steps resulting in improved comparability across different studies. The impact of these choices will be demonstrated using intrinsic (e.g.~reliability, number of properties) and extrinsic measures (e.g., categorization, semantic similarity, lexical processing). Example data and the impact of choice decisions will be provided. This practical primer will offer potential solutions to several longstanding problems and allow researchers to develop new property listing norms overcoming the constraints of previous studies.


}

\begin{document}
\maketitle

\begin{enumerate}
\def\labelenumi{\arabic{enumi}.}
\tightlist
\item
  Available feature norms and their format
\end{enumerate}

\begin{itemize}
\tightlist
\item
  Property listing task original work: Toglia and Battig (1978); Toglia (2009); Rosch and Mervis (1975); Ashcraft (1978)
\item
  English: McRae, Cree, Seidenberg, and McNorgan (2005), Vinson and Vigliocco (2008), Buchanan, Holmes, Teasley, and Hutchison (2013), Devereux, Tyler, Geertzen, and Randall (2014), Buchanan, Valentine, and Maxwell (2019)
\item
  Italian: Montefinese, Ambrosini, Fairfield, and Mammarella (2013); Reverberi, Capitani, and Laiacona (2004), Kremer and Baroni (2011)
\item
  German: Kremer and Baroni (2011)
\item
  Portuguese: Stein and de Azevedo Gomes (2009)
\item
  Spanish: Vivas, Vivas, Comesaña, Coni, and Vorano (2017)
\item
  Dutch: Ruts et al. (2004)
\item
  Blind participants: Lenci, Baroni, Cazzolli, and Marotta (2013)
\end{itemize}

I'm sure there are more, here's what we cited recently.

Define concept, feature for clarity throughout - make sure you use these two terms consistently.

\begin{enumerate}
\def\labelenumi{\arabic{enumi}.}
\setcounter{enumi}{1}
\tightlist
\item
  Pointers about how to collect the data
\end{enumerate}

\begin{enumerate}
\def\labelenumi{\alph{enumi}.}
\tightlist
\item
  instructions, generation, verification, importance
\end{enumerate}

I really like the way the CSLB did it: \url{https://cslb.psychol.cam.ac.uk/propnorms}

They showed the concept, then had a drop down menu for is/has/does, and then the participant typed in a final window. That type of system would solve about half the problems I am going to describe below about using multi-word sequences. Might be some other suggestions, but for that type of processing, you could do combinations and have more consistent data easily.

\begin{enumerate}
\def\labelenumi{\arabic{enumi}.}
\setcounter{enumi}{2}
\tightlist
\item
  Typical operations performed on features
\end{enumerate}

\begin{figure}
\includegraphics[width=5.28in]{flow_chart} \caption{Flow chart of proposed semantic processing feature steps.}\label{fig:flowchart}
\end{figure}

In the next several sections, we provide a tutorial using \emph{R} on how data from the semantic property norm task might be processed from raw input to finalized output. Figure \ref{fig:flowchart} portrays the proposed set of steps including spell checking, lemmatization, exclusion of stop words, and final processing in a multi-word sequence approach or a bag of words approach. After detailing these steps, the final data form will compared to previous norms to determine the usefulness of this approach.

\hypertarget{materials-and-data-format}{%
\subsection{Materials and Data Format}\label{materials-and-data-format}}

\begin{table}[t]

\caption{\label{tab:tab1}Example of Data Formatted for Tidy Data}
\centering
\begin{tabular}{l>{\raggedright\arraybackslash}p{30em}}
\toprule
word & answer\\
\midrule
airplane & you fly in it  its big  it is fast  they are expensive  they are at an airport  you have to be trained to fly it  there are lots of seats  they get very high up\\
airplane & wings engine pilot cockpit tail\\
airplane & wings  it flys  modern technology  has passengers  requires a pilot  can be dangerous  runs on gas  used for travel\\
airplane & wings  flys  pilot  cockpit  uses gas  faster travel\\
airplane & wings  engines  passengers  pilot(s)  vary in size and color\\
\addlinespace
airplane & wings  body  flies  travel\\
\bottomrule
\end{tabular}
\end{table}

The data for this tutorial includes 16544 unique concept-feature responses for 226 concepts from Buchanan et al. (2019) that were included in McRae et al. (2005), Vinson and Vigliocco (2008), and Bruni, Tran, and Baroni (2014). The data should be structured in tidy format wherein each concept-feature observation is a row and each column is a variable (Wickham, 2014). Therefore, the data includes a \texttt{word} column with the normed concept and an \texttt{answer} column with the participant answer, as shown in Table \ref{tab:tab1}.

This data was collected using the instructions provided by McRae et al. (2005), however, in contrast to the suggestions for consistency detailed above (Devereux et al., 2014), each participant was simply given a large text box to include their answer. Each answer includes multiple embedded features, and the tutorial proceeds to demonstrate potential processing addressing the data in this nature. With structured data entry for participants, the suggested processing steps are reduced.

\hypertarget{spelling}{%
\subsection{Spelling}\label{spelling}}

Spell checking can be automated with the \texttt{hunspell} package in \emph{R} (Ooms, 2018), which is the spell checking library used in popular programs such as FireFox, Chrome, RStudio, and OpenOffice. Each \texttt{answer} can be checked for misspellings across an entire column of answers, which is located in the \texttt{master} dataset. The default dictionary is American English, and the \texttt{hunspell} vignettes provide details on how to import your own dictionary for non-English languages. The choice of dictionary should also normalize between multiple varieties of the same language, for example, the \texttt{"en\_GB"} would convert to British English spellings.

\scriptsize

\begin{Shaded}
\begin{Highlighting}[]
\CommentTok{## Install the hunspell package if necessary}
\CommentTok{#install.packages("hunspell")}
\KeywordTok{library}\NormalTok{(hunspell)}
\CommentTok{## Check the participant answers}
\CommentTok{## The output is a list of spelling errors for each line}
\NormalTok{spelling_errors <-}\StringTok{ }\KeywordTok{hunspell}\NormalTok{(master}\OperatorTok{$}\NormalTok{answer, }\DataTypeTok{dict =} \KeywordTok{dictionary}\NormalTok{(}\StringTok{"en_US"}\NormalTok{))}
\end{Highlighting}
\end{Shaded}

\normalsize

The result from the \texttt{hunspell()} function is a list object of spelling errors for each row of data. For example, when responding to \emph{apple}, a participant wrote \emph{fruit containing seeds}, and the spelling errors were denoted as **. After checking for errors, the \texttt{hunspell\_suggest()} function was used to determine the most likely replacement for each error.

\scriptsize

\begin{Shaded}
\begin{Highlighting}[]
\CommentTok{## Check for suggestions}
\NormalTok{spelling_suggest <-}\StringTok{ }\KeywordTok{lapply}\NormalTok{(spelling_errors, hunspell_suggest)}
\end{Highlighting}
\end{Shaded}

\normalsize

For \emph{NA}, both ** were suggested, and ** were suggested for \emph{NA}. The suggestions are presented in most probable order, and using a few loops with the substitute (\texttt{gsub()}) function, we can replace all errors with the most likely replacement in a new dataset \texttt{spell\_checked}. A specialized dictionary with precoded error responses and corrections could be implemented at this stage. Other paid alternatives, such as Bing Spell Check, can be a useful avenue for datasets that may contain brand names (i.e, \emph{apple} versus \emph{Apple}) or slang terms.

\scriptsize

\begin{Shaded}
\begin{Highlighting}[]
\CommentTok{## Replace with most likely suggestion}
\NormalTok{spell_checked <-}\StringTok{ }\NormalTok{master}
\CommentTok{### Loop over the dataframe}
\ControlFlowTok{for}\NormalTok{ (i }\ControlFlowTok{in} \DecValTok{1}\OperatorTok{:}\KeywordTok{nrow}\NormalTok{(spell_checked))\{}
  \CommentTok{### See if there are spelling errors}
  \ControlFlowTok{if}\NormalTok{ (}\KeywordTok{length}\NormalTok{(spelling_errors[[i]]) }\OperatorTok{>}\StringTok{ }\DecValTok{0}\NormalTok{) \{}
    \CommentTok{### Loop over all errors}
    \ControlFlowTok{for}\NormalTok{ (q }\ControlFlowTok{in} \DecValTok{1}\OperatorTok{:}\KeywordTok{length}\NormalTok{(spelling_errors[[i]]))\{}
      \CommentTok{### Replace with the first answer}
\NormalTok{      spell_checked}\OperatorTok{$}\NormalTok{answer[i] <-}\StringTok{ }\KeywordTok{gsub}\NormalTok{(spelling_errors[[i]][q], }
\NormalTok{                                      spelling_suggest[[i]][[q]][}\DecValTok{1}\NormalTok{],}
\NormalTok{                                      spell_checked}\OperatorTok{$}\NormalTok{answer[i])}
\NormalTok{    \}}
\NormalTok{  \}}
\NormalTok{\}}
\end{Highlighting}
\end{Shaded}

\normalsize

\hypertarget{lemmatization}{%
\subsection{Lemmatization}\label{lemmatization}}

The next step approaches the clustering of word forms into their lemma or head word from a dictionary. The process of lemmatizing words involves using a lexeme set (i.e., all words forms that have the same meaning, \emph{am, are, is}) to convert into a common lemma (i.e., \emph{be}) from a trained dictionary. In contrast, stemming involves processing words using heuristics to remove affixes or inflections, such as \emph{ing} or \emph{s}. The stem or root word may not reflect an actual word in the langauge, as simply removing an affix does not necessarily produce the lemma. For example, in response to \emph{airplane}, \emph{flying} can be easily converted to \emph{fly} by removing the \emph{ing} inflection. However, this same heuristic converts the feature \emph{wings} into \emph{w} after removing both the \emph{s} for a plural marker and the \emph{ing} participle marker. Several packages for \emph{R} include customizable stemmers, notably the \texttt{hunspell}, \texttt{corpus} (Perry, 2017), and \texttt{tm} (Feinerer, Hornik, \& Artifex Software, 2018) packages.

Lemmatization is the likely choice for processing property norms, and this process can be achieved by installing \texttt{TreeTagger} (Schmid, 1994) and the \texttt{koRpus} package in \emph{R} (Michalke, 2018). TreeTagger is a trained tagger designed to annotate part of speech and lemma information in text, and parameter files are available for multiple langauges. The koRpus package includes functionality to use TreeTagger in \emph{R}. After installing the package and TreeTagger, we will create a unique set of tokenized words to lemmatize to speed computation.

\scriptsize

\begin{Shaded}
\begin{Highlighting}[]
\NormalTok{lemmas <-}\StringTok{ }\NormalTok{spell_checked}
\CommentTok{## Install the koRpus package}
\CommentTok{#install.packages("koRpus")}
\CommentTok{#install.packages("koRpus.lang.en")}
\CommentTok{## You must load both packages separately}
\KeywordTok{library}\NormalTok{(koRpus)}
\KeywordTok{library}\NormalTok{(koRpus.lang.en)}
\CommentTok{## Install TreeTagger }
\CommentTok{#https://www.cis.uni-muenchen.de/~schmid/tools/TreeTagger/}
\CommentTok{## Find all types for faster lookup}
\NormalTok{all_answers <-}\StringTok{ }\KeywordTok{tokenize}\NormalTok{(lemmas}\OperatorTok{$}\NormalTok{answer, }\DataTypeTok{format =} \StringTok{"obj"}\NormalTok{, }\DataTypeTok{tag =}\NormalTok{ F)}
\NormalTok{all_answers <-}\StringTok{ }\KeywordTok{unique}\NormalTok{(all_answers)}
\end{Highlighting}
\end{Shaded}

\normalsize

The \texttt{treetag()} function calls the installation of TreeTagger to provide part of speech tags and lemmas for each token. Importantly, the \texttt{path} option should be the directory of the TreeTagger installation.

\scriptsize

\begin{Shaded}
\begin{Highlighting}[]
\CommentTok{## This function has both suppressWarnings & suppressMessages}
\CommentTok{## You should first view these to ensure proper processing}
\NormalTok{temp_tag <-}\StringTok{ }\KeywordTok{suppressWarnings}\NormalTok{(}
  \KeywordTok{suppressMessages}\NormalTok{(}
    \CommentTok{## Note: the NULL option is to control for the <unknown> that appears}
    \CommentTok{## to occur with the last word in each text}
    \KeywordTok{treetag}\NormalTok{(}\KeywordTok{c}\NormalTok{(all_answers, }\StringTok{"NULL"}\NormalTok{), }
            \CommentTok{## Control the parameters of treetagger}
            \DataTypeTok{treetagger=}\StringTok{"manual"}\NormalTok{, }\DataTypeTok{format=}\StringTok{"obj"}\NormalTok{,}
            \DataTypeTok{TT.tknz=}\OtherTok{FALSE}\NormalTok{, }\DataTypeTok{lang=}\StringTok{"en"}\NormalTok{,}
            \DataTypeTok{TT.options=}\KeywordTok{list}\NormalTok{(}\DataTypeTok{path=}\StringTok{"~/TreeTagger"}\NormalTok{, }\DataTypeTok{preset=}\StringTok{"en"}\NormalTok{))))}
\end{Highlighting}
\end{Shaded}

\normalsize

This function returns a tagged corpus object, which can be converted into a dataframe of the token-lemma information. The goal would be to replace inflected words with their lemmas, and therefore, unknown values, number tags, and equivalent values are ignored by subseting out these from the dataset. Table \ref{tab:tab2} portrays the results from TreeTagger.

\scriptsize

\begin{Shaded}
\begin{Highlighting}[]
\CommentTok{## Remove all tags not using}
\NormalTok{replacement_lemmas <-}\StringTok{ }\NormalTok{temp_tag}\OperatorTok{@}\NormalTok{TT.res}
\NormalTok{replacement_lemmas <-}\StringTok{ }\KeywordTok{subset}\NormalTok{(replacement_lemmas, }
                             \CommentTok{#ignore punctuation}
\NormalTok{                             wclass }\OperatorTok{!=}\StringTok{ "punctuation"} \OperatorTok{&}
\StringTok{                             }\CommentTok{#unknown values}
\StringTok{                             }\NormalTok{lemma }\OperatorTok{!=}\StringTok{ "<unknown>"} \OperatorTok{&}\StringTok{ }
\StringTok{                             }\CommentTok{#numbers}
\StringTok{                             }\NormalTok{lemma}\OperatorTok{!=}\StringTok{ "@card@"} \OperatorTok{&}\StringTok{ }
\StringTok{                             }\CommentTok{#token should change more than case}
\StringTok{                             }\KeywordTok{tolower}\NormalTok{(token) }\OperatorTok{!=}\StringTok{ }\KeywordTok{tolower}\NormalTok{(lemma)) }
\end{Highlighting}
\end{Shaded}

\normalsize

\begin{table}[t]

\caption{\label{tab:tab2}Lemma and Part of Speech Information from TreeTagger}
\centering
\begin{tabular}{lllrl}
\toprule
token & tag & lemma & lttr & wclass\\
\midrule
is & VBZ & be & 2 & verb\\
are & VBP & be & 3 & verb\\
trained & VBN & train & 7 & verb\\
lots & NNS & lot & 4 & noun\\
seats & NNS & seat & 5 & noun\\
\addlinespace
wings & NNS & wing & 5 & noun\\
\bottomrule
\end{tabular}
\end{table}

From this dataset, you can use the \texttt{stringi} package (Gagolewski \& Tartanus, 2019) to replace all of the original tokens with their lemmas. This package allows for replacement lookup across a large set of subsitutions. The \texttt{stri\_replace\_all\_regex()} function includes the column of data to examine, the patterns to find (using \texttt{\textbackslash{}\textbackslash{}b} regular expressions to ensure word boundaries and no partial word replacements), what to replace those patterns with, and other options to ensure the original dataframe with replacement is returned. Table \ref{tab:tab3} shows the processed data at this stage.

\scriptsize

\begin{Shaded}
\begin{Highlighting}[]
\CommentTok{## Install the stringi package}
\CommentTok{#install.packages("stringi")}
\KeywordTok{library}\NormalTok{(stringi)}
\CommentTok{## Replace all the original tokens with new lemmas using \textbackslash{}\textbackslash{}b for word boundaries}
\NormalTok{lemmas}\OperatorTok{$}\NormalTok{answer <-}\StringTok{ }\KeywordTok{stri_replace_all_regex}\NormalTok{(}\DataTypeTok{str =}\NormalTok{ lemmas}\OperatorTok{$}\NormalTok{answer, }
                       \DataTypeTok{pattern =} \KeywordTok{paste}\NormalTok{(}\StringTok{"}\CharTok{\textbackslash{}\textbackslash{}}\StringTok{b"}\NormalTok{, replacement_lemmas}\OperatorTok{$}\NormalTok{token, }\StringTok{"}\CharTok{\textbackslash{}\textbackslash{}}\StringTok{b"}\NormalTok{, }\DataTypeTok{sep =} \StringTok{""}\NormalTok{),}
                       \DataTypeTok{replacement =}\NormalTok{ replacement_lemmas}\OperatorTok{$}\NormalTok{lemma,}
                       \DataTypeTok{vectorize_all =}\NormalTok{ F, }\KeywordTok{list}\NormalTok{(}\DataTypeTok{case_insensitive =} \OtherTok{TRUE}\NormalTok{))}
\end{Highlighting}
\end{Shaded}

\normalsize

\begin{table}[t]

\caption{\label{tab:tab3}Original Data with Lemmatization}
\centering
\begin{tabular}{l>{\raggedright\arraybackslash}p{30em}}
\toprule
word & answer\\
\midrule
airplane & you fly in it  its big  it be fast  they be expensive  they be at an airport  you have to be train to fly it  there be lot of seat  they get very high up\\
airplane & wing engine pilot cockpit tail\\
airplane & wing  it fly  modern technology  have passenger  require a pilot  can be dangerous  run on gas  use for travel\\
airplane & wing  fly  pilot  cockpit  use gas  fast travel\\
airplane & wing  engine  passenger  pilot(s)  vary in size and color\\
\addlinespace
airplane & wing  body  fly  travel\\
\bottomrule
\end{tabular}
\end{table}

\hypertarget{word-sequences}{%
\subsection{Word Sequences}\label{word-sequences}}

Multi-word sequences are often coded to mimic a Collins and Quillian (1969) style model, with \enquote{is-a} and \enquote{has-a} type markers. If data were collected to include these markers, this step would be pre-encoded into the output data, rendering the following code unnecessary. A potential solution for processing messy data could be to search for specific part of speech sequences that mimic the \enquote{is-a} and \enquote{has-a} strings. An examination of the coding in McRae et al. (2005) and Devereux et al. (2014) indicates that the feature tags are often verb-noun or verb-adjective-noun sequences. Using TreeTagger on each concept's answer set, we can obtain the parts of speech in context for each lemma. With \texttt{dplyr} (Wickham, Francios, Henry, Muller, \& Rstudio, 2019), new columns are added to tagged data to show all bigram and trigram sequences. All verb-noun and verb-adjective-noun combinations are selected, and any words not part of these multi-word sequences are treated as unigrams. Finally, the \texttt{table()} function is used to tabulate the final count of n-grams and their frequency.

\scriptsize

\begin{Shaded}
\begin{Highlighting}[]
\CommentTok{## Create an empty dataframe }
\NormalTok{multi_words <-}\StringTok{ }\KeywordTok{data.frame}\NormalTok{(}\DataTypeTok{Word=}\KeywordTok{character}\NormalTok{(),}
                        \DataTypeTok{Feature=}\KeywordTok{character}\NormalTok{(), }
                        \DataTypeTok{Frequency=}\KeywordTok{numeric}\NormalTok{(), }
                        \DataTypeTok{stringsAsFactors=}\OtherTok{FALSE}\NormalTok{) }
\CommentTok{## Create unique word list to loop over }
\NormalTok{unique_concepts <-}\StringTok{ }\KeywordTok{unique}\NormalTok{(lemmas}\OperatorTok{$}\NormalTok{word)}
\CommentTok{## Install dplyr}
\CommentTok{#install.packages("dplyr")}
\KeywordTok{library}\NormalTok{(dplyr)}
\CommentTok{## Loop over each word}
\ControlFlowTok{for}\NormalTok{ (i }\ControlFlowTok{in} \DecValTok{1}\OperatorTok{:}\KeywordTok{length}\NormalTok{(unique_concepts))\{}
  \CommentTok{## Create parts of speech for clustering together}
\NormalTok{  temp_tag <-}\StringTok{ }\KeywordTok{suppressWarnings}\NormalTok{(}
    \KeywordTok{suppressMessages}\NormalTok{(}
      \KeywordTok{treetag}\NormalTok{(}\KeywordTok{c}\NormalTok{(lemmas}\OperatorTok{$}\NormalTok{answer[lemmas}\OperatorTok{$}\NormalTok{word  }\OperatorTok{==}\StringTok{ }\NormalTok{unique_concepts[i]], }\StringTok{"NULL"}\NormalTok{), }
          \CommentTok{## Control the parameters of treetagger}
          \DataTypeTok{treetagger=}\StringTok{"manual"}\NormalTok{, }\DataTypeTok{format=}\StringTok{"obj"}\NormalTok{,}
          \DataTypeTok{TT.tknz=}\OtherTok{FALSE}\NormalTok{, }\DataTypeTok{lang=}\StringTok{"en"}\NormalTok{,}
          \DataTypeTok{TT.options=}\KeywordTok{list}\NormalTok{(}\DataTypeTok{path=}\StringTok{"~/TreeTagger"}\NormalTok{, }\DataTypeTok{preset=}\StringTok{"en"}\NormalTok{))))}
  \CommentTok{## Save only the dataframe, remove NULL}
\NormalTok{  temp_tag <-}\StringTok{ }\NormalTok{temp_tag}\OperatorTok{@}\NormalTok{TT.res[}\OperatorTok{-}\KeywordTok{nrow}\NormalTok{(temp_tag}\OperatorTok{@}\NormalTok{TT.res) , ]}
  \CommentTok{## Subset out information you don't need}
\NormalTok{  temp_tag <-}\StringTok{ }\KeywordTok{subset}\NormalTok{(temp_tag, }
\NormalTok{                     wclass }\OperatorTok{!=}\StringTok{ "comma"} \OperatorTok{&}\StringTok{ }\NormalTok{wclass }\OperatorTok{!=}\StringTok{ "determiner"} \OperatorTok{&}\StringTok{ }
\StringTok{                       }\NormalTok{wclass }\OperatorTok{!=}\StringTok{ "preposition"} \OperatorTok{&}\StringTok{ }\NormalTok{wclass }\OperatorTok{!=}\StringTok{ "modal"} \OperatorTok{&}
\StringTok{                       }\NormalTok{wclass }\OperatorTok{!=}\StringTok{ "predeterminer"} \OperatorTok{&}\StringTok{ }\NormalTok{wclass }\OperatorTok{!=}\StringTok{ "particle"} \OperatorTok{&}
\StringTok{                       }\NormalTok{wclass }\OperatorTok{!=}\StringTok{ "to"} \OperatorTok{&}\StringTok{ }\NormalTok{wclass }\OperatorTok{!=}\StringTok{ "punctuation"} \OperatorTok{&}\StringTok{ }
\StringTok{                       }\NormalTok{wclass }\OperatorTok{!=}\StringTok{ "fullstop"} \OperatorTok{&}\StringTok{ }\NormalTok{wclass }\OperatorTok{!=}\StringTok{ "conjunction"} \OperatorTok{&}\StringTok{ }
\StringTok{                       }\NormalTok{wclass }\OperatorTok{!=}\StringTok{ "pronoun"}\NormalTok{)}
  \CommentTok{## Create a temporary tibble }
\NormalTok{  temp_tag_tibble <-}\StringTok{ }\KeywordTok{as_tibble}\NormalTok{(temp_tag)}
  \CommentTok{## Create part of speech and features combined}
\NormalTok{  temp_tag_tibble <-}\StringTok{ }\KeywordTok{mutate}\NormalTok{(temp_tag_tibble, }
                            \DataTypeTok{two_words =} \KeywordTok{paste}\NormalTok{(token, }
                                              \KeywordTok{lead}\NormalTok{(token), }\DataTypeTok{sep =} \StringTok{"_"}\NormalTok{))}
\NormalTok{  temp_tag_tibble <-}\StringTok{ }\KeywordTok{mutate}\NormalTok{(temp_tag_tibble, }
                            \DataTypeTok{three_words =} \KeywordTok{paste}\NormalTok{(token, }
                                                \KeywordTok{lead}\NormalTok{(token), }\KeywordTok{lead}\NormalTok{(token, }\DataTypeTok{n =}\NormalTok{ 2L), }
                                                \DataTypeTok{sep =} \StringTok{"_"}\NormalTok{))}
\NormalTok{  temp_tag_tibble <-}\StringTok{ }\KeywordTok{mutate}\NormalTok{(temp_tag_tibble, }
                            \DataTypeTok{two_words_pos =} \KeywordTok{paste}\NormalTok{(wclass, }
                                                  \KeywordTok{lead}\NormalTok{(wclass), }\DataTypeTok{sep =} \StringTok{"_"}\NormalTok{))}
\NormalTok{  temp_tag_tibble <-}\StringTok{ }\KeywordTok{mutate}\NormalTok{(temp_tag_tibble, }
                            \DataTypeTok{three_words_pos =} \KeywordTok{paste}\NormalTok{(wclass, }
                                                    \KeywordTok{lead}\NormalTok{(wclass), }\KeywordTok{lead}\NormalTok{(wclass, }\DataTypeTok{n =}\NormalTok{ 2L), }
                                                    \DataTypeTok{sep =} \StringTok{"_"}\NormalTok{))}
  \CommentTok{## Find verb noun or verb adjective nouns to cluster on }
\NormalTok{  verb_nouns <-}\StringTok{ }\KeywordTok{grep}\NormalTok{(}\StringTok{"}\CharTok{\textbackslash{}\textbackslash{}}\StringTok{bverb_noun"}\NormalTok{, temp_tag_tibble}\OperatorTok{$}\NormalTok{two_words_pos)}
\NormalTok{  verb_adj_nouns <-}\StringTok{ }\KeywordTok{grep}\NormalTok{(}\StringTok{"}\CharTok{\textbackslash{}\textbackslash{}}\StringTok{bverb_adjective_noun"}\NormalTok{, temp_tag_tibble}\OperatorTok{$}\NormalTok{three_words_pos)}
  \CommentTok{## Use combined and left over features}
\NormalTok{  features_for_table <-}\StringTok{ }\KeywordTok{c}\NormalTok{(temp_tag_tibble}\OperatorTok{$}\NormalTok{two_words[verb_nouns], }
\NormalTok{                          temp_tag_tibble}\OperatorTok{$}\NormalTok{three_words[verb_adj_nouns],}
\NormalTok{                          temp_tag_tibble}\OperatorTok{$}\NormalTok{token[}\OperatorTok{-}\KeywordTok{c}\NormalTok{(verb_nouns, verb_nouns}\OperatorTok{+}\DecValTok{1}\NormalTok{, }
\NormalTok{                                                   verb_adj_nouns, verb_adj_nouns}\OperatorTok{+}\DecValTok{1}\NormalTok{, }
\NormalTok{                                                   verb_adj_nouns}\OperatorTok{+}\DecValTok{2}\NormalTok{)])}
  \CommentTok{## Create a table of frequencies}
\NormalTok{  word_table <-}\StringTok{ }\KeywordTok{as.data.frame}\NormalTok{(}\KeywordTok{table}\NormalTok{(features_for_table))}
  \CommentTok{## Clean up the table}
\NormalTok{  word_table}\OperatorTok{$}\NormalTok{Word <-}\StringTok{ }\NormalTok{unique_concepts[i]}
  \KeywordTok{colnames}\NormalTok{(word_table) =}\StringTok{ }\KeywordTok{c}\NormalTok{(}\StringTok{"Feature"}\NormalTok{, }\StringTok{"Frequency"}\NormalTok{, }\StringTok{"Word"}\NormalTok{)}
\NormalTok{  multi_words <-}\StringTok{ }\KeywordTok{rbind}\NormalTok{(multi_words, word_table[ , }\KeywordTok{c}\NormalTok{(}\DecValTok{3}\NormalTok{, }\DecValTok{1}\NormalTok{, }\DecValTok{2}\NormalTok{)])}
\NormalTok{\}}
\end{Highlighting}
\end{Shaded}

\normalsize

This procedure produces mostly positive output, such as \emph{fingers-have\_fingernails} and \emph{couches-have\_cushions}. One obvious limitation is the potential necessity to match this coding system to previous codes, which were predominately hand processed. Further, many similar phrases, such as the ones for \emph{zebra} shown below may require fuzzy logic matching to ensure that the different codings for \emph{is-a-horse} are all combined together, as shown in Table \ref{tab:tab4}.

\begin{table}[t]

\caption{\label{tab:tab4}Multi-Word Sequence Examples for Zebra}
\centering
\begin{tabular}{llr}
\toprule
Word & Feature & Frequency\\
\midrule
zebra & be\_horse & 1\\
zebra & be\_similar\_horse & 1\\
zebra & build\_horse & 1\\
zebra & horse & 22\\
zebra & horse-like & 1\\
\addlinespace
zebra & look\_similar\_horse & 1\\
zebra & related\_horse & 1\\
zebra & resemble\_small\_horse & 1\\
zebra & run\_fast\_horse & 1\\
zebra & run\_horse & 1\\
\bottomrule
\end{tabular}
\end{table}

\hypertarget{bag-of-words}{%
\subsection{Bag of Words}\label{bag-of-words}}

The bag of words approach simply treats each token as a separate feature to be tabulated for analysis. After stemming and lemmatization, the data can be processed as single word tokens into a table of frequencies for each cue word. The resulting dataframe is each cue-feature combination with a total for each feature.

\scriptsize

\begin{Shaded}
\begin{Highlighting}[]
\CommentTok{## Create an empty dataframe }
\NormalTok{bag_words <-}\StringTok{ }\KeywordTok{data.frame}\NormalTok{(}\DataTypeTok{Word=}\KeywordTok{character}\NormalTok{(),}
                        \DataTypeTok{Feature=}\KeywordTok{character}\NormalTok{(), }
                        \DataTypeTok{Frequency=}\KeywordTok{numeric}\NormalTok{(), }
                        \DataTypeTok{stringsAsFactors=}\OtherTok{FALSE}\NormalTok{) }
\CommentTok{## Loop over each word}
\ControlFlowTok{for}\NormalTok{ (i }\ControlFlowTok{in} \DecValTok{1}\OperatorTok{:}\KeywordTok{length}\NormalTok{(unique_concepts))\{}
  \CommentTok{## Create a table of frequencies}
\NormalTok{  word_table <-}\StringTok{ }\KeywordTok{as.data.frame}\NormalTok{(}\KeywordTok{table}\NormalTok{(}
    \CommentTok{## Tokenize the words}
    \KeywordTok{tokenize}\NormalTok{(}
      \CommentTok{## Put all answers together in one character string}
      \KeywordTok{paste0}\NormalTok{(lemmas}\OperatorTok{$}\NormalTok{answer[lemmas}\OperatorTok{$}\NormalTok{word }\OperatorTok{==}\StringTok{ }\NormalTok{unique_concepts[i]], }\DataTypeTok{collapse =} \StringTok{" "}\NormalTok{), }
      \DataTypeTok{format =} \StringTok{"obj"}\NormalTok{, }\DataTypeTok{tag =}\NormalTok{ F)))}
  
  \CommentTok{## Clean up the table}
\NormalTok{  word_table}\OperatorTok{$}\NormalTok{Word <-}\StringTok{ }\NormalTok{unique_concepts[i]}
  \KeywordTok{colnames}\NormalTok{(word_table) =}\StringTok{ }\KeywordTok{c}\NormalTok{(}\StringTok{"Feature"}\NormalTok{, }\StringTok{"Frequency"}\NormalTok{, }\StringTok{"Word"}\NormalTok{)}
  
\NormalTok{  bag_words <-}\StringTok{ }\KeywordTok{rbind}\NormalTok{(bag_words, word_table[ , }\KeywordTok{c}\NormalTok{(}\DecValTok{3}\NormalTok{, }\DecValTok{1}\NormalTok{, }\DecValTok{2}\NormalTok{)])}
\NormalTok{\}}
\CommentTok{## Remove punctuation}
\NormalTok{bag_words <-}\StringTok{ }\NormalTok{bag_words[}\OperatorTok{-}\KeywordTok{c}\NormalTok{(}\KeywordTok{grep}\NormalTok{(}\StringTok{'^[[:punct:]]'}\NormalTok{,bag_words}\OperatorTok{$}\NormalTok{Feature)), ]}
\end{Highlighting}
\end{Shaded}

\normalsize

Tab @\ref(tab:tab5) shows the top ten most frequent responses to \emph{zebra} given the bag of words approach. The top ten features in zebra indicate a match to the multi-word sequence approach but the inclusion of words such as \emph{be, in, a} indicate the need to remove irrelevant words listed with features.

\begin{table}[t]

\caption{\label{tab:tab5}Bag of Words Examples for Zebra}
\centering
\begin{tabular}{llr}
\toprule
Word & Feature & Frequency\\
\midrule
zebra & stripe & 71\\
zebra & black & 63\\
zebra & white & 61\\
zebra & be & 56\\
zebra & animal & 54\\
\addlinespace
zebra & have & 54\\
zebra & a & 46\\
zebra & and & 46\\
zebra & in & 41\\
zebra & horse & 32\\
\bottomrule
\end{tabular}
\end{table}

\hypertarget{stopwords}{%
\subsection{Stopwords}\label{stopwords}}

As shown in Figure \ref{fig:flowchart}, the next stage of processing would be to exclude stopwords, such as \emph{the, of, but}, for either the multi-word sequence or bag of word style processing. The \texttt{stopwords} package ({\textbf{???}}) includes a list of stopwords for more than 50 languages. For multi-word sequence processing, these values can be removed by subseting the data to exclude stopwords as unigrams.

\scriptsize

\begin{Shaded}
\begin{Highlighting}[]
\CommentTok{## Install the stopwords package or use tm}
\CommentTok{#install.packages("stopwords")}
\KeywordTok{library}\NormalTok{(stopwords)}
\CommentTok{## Remove stop words from either processing approach}
\NormalTok{multi_words <-}\StringTok{ }\KeywordTok{subset}\NormalTok{(multi_words, }
                      \OperatorTok{!}\NormalTok{(Feature }\OperatorTok\StringTok{ }\KeywordTok{stopwords}\NormalTok{(}\DataTypeTok{language =} \StringTok{"en"}\NormalTok{, }
                                               \DataTypeTok{source =} \StringTok{"snowball"}\NormalTok{)))}

\NormalTok{bag_words <-}\StringTok{ }\KeywordTok{subset}\NormalTok{(bag_words, }
                    \OperatorTok{!}\NormalTok{(Feature }\OperatorTok\StringTok{ }\KeywordTok{stopwords}\NormalTok{(}\DataTypeTok{language =} \StringTok{"en"}\NormalTok{, }
                                             \DataTypeTok{source =} \StringTok{"snowball"}\NormalTok{)))}
\end{Highlighting}
\end{Shaded}

\normalsize

\hypertarget{descriptive-statistics}{%
\subsection{Descriptive Statistics}\label{descriptive-statistics}}

The finalized data now represents a a processed set of cue-feature combinations with their frequencies for analysis. Given the differences in sample size across data collection points from Buchanan et al. (2019), this information was merged with the sample data. Table @\ref(tab:tab6) includes descriptive statistics for the processed cue-feature set. First, the number of cue-feature combinations was calculated by taking the average number of cue-feature listings for each cue. Therefore, the total number of features listed for \emph{zebra} might be 100, while \emph{apple} might be 45, and these values were averaged.

More cue-feature combinations are listed for the multi-word approach, likely due to differences in combinations for some overlapping features as shown in Table \ref{tab:tab4}. The large standard deviation for both approaches indicates that cues have a wide range of possible features listed. The correlation provided represents the relation between sample size for a cue and the number of features listed for that cue. These values are high and positive, indicating that the number of unique features increases with each participant. Potentially, many of the cue-feature combinations could be considered idiosyncratic. The next row of the table denotes the average number of cue-feature responses listed by less than 10\% of the participants. This porportion of responses is somewhat arbitrary, as each researcher has determined where the optimal criterion should be. For example, McRae et al. (2005) used 16\% or 5/30 participants as a minimum standard, and Buchanan et al. (2019) recently used a similar criteria. The average number of cue-features that would be considered low in proportion is quite large, indicating that these are potentially idiosyncratic or part of long tailed distribution of feature responses with many low frequency features. The advantage to the suggested data processing pipeline and code provided here is the ability of each researcher to determine their own level of response necessary, if desired.

\begin{verbatim}
## [1] 0.9836955
\end{verbatim}

\begin{verbatim}
## [1] 0.7431313
\end{verbatim}

\begin{verbatim}
## [1] 0.6645615
\end{verbatim}

\begin{verbatim}
## [1] 0.9124089
\end{verbatim}

\begin{verbatim}
## [1] 0.7281558
\end{verbatim}

\begin{verbatim}
## [1] 0.8262608
\end{verbatim}

\begin{table}[t]

\caption{\label{tab:tab6}Descriptive Statistics of Text Processing Style}
\centering
\begin{tabular}{lllllll}
\toprule
\multicolumn{1}{c}{ } & \multicolumn{3}{c}{Multi-Word Sequences} & \multicolumn{3}{c}{Bag of Words} \\
\cmidrule(l{3pt}r{3pt}){2-4} \cmidrule(l{3pt}r{3pt}){5-7}
Statistics & $M$ & $SD$ & $r$ & $M$ & $SD$ & $r$\\
\midrule
Number of Cue-Features & 191.85 & 98.19 & 0.74 & 171.80 & 76.96 & 0.66\\
Frequency of Idiosyncratic Response & 182.57 & 96.43 & 0.76 & 158.85 & 73.97 & 0.69\\
Frequency of Cue-Feature Response & 2.14 & 3.46 & 0.73 & 2.73 & 4.80 & 0.83\\
Proportion of Cue-Feature Response & 3.47 & 5.14 & -0.64 & 4.34 & 4.80 & -0.62\\
\bottomrule
\multicolumn{7}{l}{\textsuperscript{} $Note$. Correlation represents the relation between the statistic listed for that row and the}\\
\multicolumn{7}{l}{sample size for the cue.}\\
\end{tabular}
\end{table}

make a table here of the stuff
talk about deleting low features or not
d. identify cut off for idiosyncratic features (should it be necessary?)

\hypertarget{internal-comparison-of-approach}{%
\subsection{Internal Comparison of Approach}\label{internal-comparison-of-approach}}

Compare this data processing to hand processed data from B2019

\begin{verbatim}
##        raw_b        raw_m        raw_v translated_b translated_m 
##    0.6871319    0.3785460    0.5924934    0.7217544    0.5782025 
## translated_v 
##    0.5822144
\end{verbatim}

\hypertarget{external-comparison-of-approach}{%
\subsection{External Comparison of Approach}\label{external-comparison-of-approach}}

Compare to the MEN dataset

\begin{verbatim}
## [1] 0.6935202
\end{verbatim}

\hypertarget{ontology-and-categorization}{%
\subsection{Ontology and Categorization}\label{ontology-and-categorization}}

\begin{verbatim}
##  [1] 0.9645379 0.8607603 0.8439264 0.8441948 0.8442215 0.8182595 0.8187568
##  [8] 0.8192547 0.8194990 0.8195310 0.7971676 0.7972675 0.7972554 0.7976975
## [15] 0.7983826 0.7995112 0.8012690 0.8015626 0.8017586 0.8020315 0.8021860
## [22] 0.7883631 0.7885703 0.7887004 0.7887905 0.7725361 0.7727847 0.7729023
## [29] 0.7732403 0.7735866 0.7737093 0.7726901 0.7727843
\end{verbatim}

\begin{verbatim}
## 
##    1    2    3    4    5    6 
## 8545  101   29    2    3    1
\end{verbatim}

\includegraphics{flt_manuscript_files/figure-latex/categorization-1.pdf}

\begin{verbatim}
##           1           2           3           4           5           6 
##    9.771328  200.712871  145.931034 2127.500000  302.666667  445.000000
\end{verbatim}

\begin{verbatim}
##         1         2         3         4         5         6 
##  26.54329 203.79776 140.18889 163.34167  80.68664        NA
\end{verbatim}

\begin{verbatim}
##    1    2    3    4    5    6 
##    1   20   24 2012  219  445
\end{verbatim}

\begin{verbatim}
##    1    2    3    4    5    6 
##  774 1239  537 2243  380  445
\end{verbatim}

\hypertarget{discussion}{%
\section{Discussion}\label{discussion}}

\newpage

\hypertarget{references}{%
\section{References}\label{references}}

\begingroup
\setlength{\parindent}{-0.5in}
\setlength{\leftskip}{0.5in}

\hypertarget{refs}{}
\leavevmode\hypertarget{ref-Ashcraft1978a}{}%
Ashcraft, M. H. (1978). Property norms for typical and atypical items from 17 categories: A description and discussion. \emph{Memory \& Cognition}, \emph{6}(3), 227--232. doi:\href{https://doi.org/10.3758/BF03197450}{10.3758/BF03197450}

\leavevmode\hypertarget{ref-Bruni2014}{}%
Bruni, E., Tran, N. K., \& Baroni, M. (2014). Multimodal Distributional Semantics. \emph{Journal of Artificial Intelligence Research}, \emph{49}, 1--47. doi:\href{https://doi.org/10.1613/jair.4135}{10.1613/jair.4135}

\leavevmode\hypertarget{ref-Buchanan2013}{}%
Buchanan, E. M., Holmes, J. L., Teasley, M. L., \& Hutchison, K. A. (2013). English semantic word-pair norms and a searchable Web portal for experimental stimulus creation. \emph{Behavior Research Methods}, \emph{45}(3), 746--757. doi:\href{https://doi.org/10.3758/s13428-012-0284-z}{10.3758/s13428-012-0284-z}

\leavevmode\hypertarget{ref-Buchanan2019}{}%
Buchanan, E. M., Valentine, K. D., \& Maxwell, N. P. (2019). English semantic feature production norms: An extended database of 4436 concepts. \emph{Behavior Research Methods}. doi:\href{https://doi.org/10.3758/s13428-019-01243-z}{10.3758/s13428-019-01243-z}

\leavevmode\hypertarget{ref-Collins1969}{}%
Collins, A. M., \& Quillian, M. R. (1969). Retrieval time from semantic memory. \emph{Journal of Verbal Learning and Verbal Behavior}, \emph{8}(2), 240--247. doi:\href{https://doi.org/10.1016/S0022-5371(69)80069-1}{10.1016/S0022-5371(69)80069-1}

\leavevmode\hypertarget{ref-Devereux2014}{}%
Devereux, B. J., Tyler, L. K., Geertzen, J., \& Randall, B. (2014). The Centre for Speech, Language and the Brain (CSLB) concept property norms. \emph{Behavior Research Methods}, \emph{46}(4), 1119--1127. doi:\href{https://doi.org/10.3758/s13428-013-0420-4}{10.3758/s13428-013-0420-4}

\leavevmode\hypertarget{ref-Feinerer2018}{}%
Feinerer, I., Hornik, K., \& Artifex Software, I. (2018). tm: Text Mining Package. Retrieved from \url{https://cran.r-project.org/web/packages/tm/index.html}

\leavevmode\hypertarget{ref-Gagolewski2019}{}%
Gagolewski, M., \& Tartanus, B. (2019). stringi: Character String Processing Facilities. Retrieved from \url{https://cran.r-project.org/web/packages/stringi/index.html}

\leavevmode\hypertarget{ref-Kremer2011a}{}%
Kremer, G., \& Baroni, M. (2011). A set of semantic norms for German and Italian. \emph{Behavior Research Methods}, \emph{43}(1), 97--109. doi:\href{https://doi.org/10.3758/s13428-010-0028-x}{10.3758/s13428-010-0028-x}

\leavevmode\hypertarget{ref-Lenci2013}{}%
Lenci, A., Baroni, M., Cazzolli, G., \& Marotta, G. (2013). BLIND: A set of semantic feature norms from the congenitally blind. \emph{Behavior Research Methods}, \emph{45}(4), 1218--1233. doi:\href{https://doi.org/10.3758/s13428-013-0323-4}{10.3758/s13428-013-0323-4}

\leavevmode\hypertarget{ref-McRae2005}{}%
McRae, K., Cree, G. S., Seidenberg, M. S., \& McNorgan, C. (2005). Semantic feature production norms for a large set of living and nonliving things. \emph{Behavior Research Methods}, \emph{37}(4), 547--559. doi:\href{https://doi.org/10.3758/BF03192726}{10.3758/BF03192726}

\leavevmode\hypertarget{ref-Michalke2018}{}%
Michalke, M. (2018). koRpus: An R Package for Text Analysis. Retrieved from \url{https://cran.r-project.org/web/packages/koRpus/index.html}

\leavevmode\hypertarget{ref-Montefinese2013}{}%
Montefinese, M., Ambrosini, E., Fairfield, B., \& Mammarella, N. (2013). Semantic memory: A feature-based analysis and new norms for Italian. \emph{Behavior Research Methods}, \emph{45}(2), 440--461. doi:\href{https://doi.org/10.3758/s13428-012-0263-4}{10.3758/s13428-012-0263-4}

\leavevmode\hypertarget{ref-Ooms2018}{}%
Ooms, J. (2018). The hunspell package: High-Performance Stemmer, Tokenizer, and Spell Checker for R. Retrieved from \href{https://cran.r-project.org/web/packages/hunspell/vignettes/intro.html\%7B/\#\%7Dsetting\%7B/_\%7Da\%7B/_\%7Dlanguage}{https://cran.r-project.org/web/packages/hunspell/vignettes/intro.html\{\textbackslash{}\#\}setting\{\textbackslash{}\_\}a\{\textbackslash{}\_\}language}

\leavevmode\hypertarget{ref-Perry2017}{}%
Perry, P. O. (2017). corpus: Text Corpus Analysis. Retrieved from \url{http://corpustext.com/}

\leavevmode\hypertarget{ref-Reverberi2004}{}%
Reverberi, C., Capitani, E., \& Laiacona, E. (2004). Variabili semantico lessicali relative a tutti gli elementi di una categoria semantica: Indagine su soggetti normali italiani per la categoria ``frutta". \emph{Giornale Italiano Di Psicologia}, \emph{31}, 497--522.

\leavevmode\hypertarget{ref-Rosch1975}{}%
Rosch, E., \& Mervis, C. B. (1975). Family resemblances: Studies in the internal structure of categories. \emph{Cognitive Psychology}, \emph{7}(4), 573--605. doi:\href{https://doi.org/10.1016/0010-0285(75)90024-9}{10.1016/0010-0285(75)90024-9}

\leavevmode\hypertarget{ref-Ruts2004}{}%
Ruts, W., De Deyne, S., Ameel, E., Vanpaemel, W., Verbeemen, T., \& Storms, G. (2004). Dutch norm data for 13 semantic categories and 338 exemplars. \emph{Behavior Research Methods, Instruments, \& Computers}, \emph{36}(3), 506--515. doi:\href{https://doi.org/10.3758/BF03195597}{10.3758/BF03195597}

\leavevmode\hypertarget{ref-Schmid1994}{}%
Schmid, H. (1994). Probabilistic Part-of-Speech Tagging Using Decision Trees. doi:\href{https://doi.org/10.1.1.28.1139}{10.1.1.28.1139}

\leavevmode\hypertarget{ref-Stein2009}{}%
Stein, L., \& de Azevedo Gomes, C. (2009). Normas Brasileiras para listas de palavras associadas: Associação semântica, concretude, frequência e emocionalidade. \emph{Psicologia: Teoria E Pesquisa}, \emph{25}, 537--546. doi:\href{https://doi.org/10.1590/S0102-37722009000400009}{10.1590/S0102-37722009000400009}

\leavevmode\hypertarget{ref-Toglia2009}{}%
Toglia, M. P. (2009). Withstanding the test of time: The 1978 semantic word norms. \emph{Behavior Research Methods}, \emph{41}(2), 531--533. doi:\href{https://doi.org/10.3758/BRM.41.2.531}{10.3758/BRM.41.2.531}

\leavevmode\hypertarget{ref-Toglia1978}{}%
Toglia, M. P., \& Battig, W. F. (1978). \emph{Handbook of semantic word norms}. Hillside, NJ: Earlbaum.

\leavevmode\hypertarget{ref-Vinson2008}{}%
Vinson, D. P., \& Vigliocco, G. (2008). Semantic feature production norms for a large set of objects and events. \emph{Behavior Research Methods}, \emph{40}(1), 183--190. doi:\href{https://doi.org/10.3758/BRM.40.1.183}{10.3758/BRM.40.1.183}

\leavevmode\hypertarget{ref-Vivas2017}{}%
Vivas, J., Vivas, L., Comesaña, A., Coni, A. G., \& Vorano, A. (2017). Spanish semantic feature production norms for 400 concrete concepts. \emph{Behavior Research Methods}, \emph{49}(3), 1095--1106. doi:\href{https://doi.org/10.3758/s13428-016-0777-2}{10.3758/s13428-016-0777-2}

\leavevmode\hypertarget{ref-Wickham2014}{}%
Wickham, H. (2014). Tidy Data. \emph{Journal of Statistical Software}, \emph{59}(10), 1--23. doi:\href{https://doi.org/10.18637/jss.v059.i10}{10.18637/jss.v059.i10}

\leavevmode\hypertarget{ref-Wickham2019}{}%
Wickham, H., Francios, R., Henry, L., Muller, K., \& Rstudio. (2019). dplyr: A Grammar of Data Manipulation. Retrieved from \url{https://cloud.r-project.org/web/packages/dplyr/index.html}

\endgroup


\end{document}
